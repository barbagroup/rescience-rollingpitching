% DO NOT EDIT - automatically generated from metadata.yaml

\def \codeURL{https://github.com/barbagroup/petibm-rollingpitching}
\def \codeDOI{}
\def \codeSWH{}
\def \dataURL{}
\def \dataDOI{}
\def \editorNAME{}
\def \editorORCID{}
\def \reviewerINAME{}
\def \reviewerIORCID{}
\def \reviewerIINAME{}
\def \reviewerIIORCID{}
\def \dateRECEIVED{02 May 2021}
\def \dateACCEPTED{}
\def \datePUBLISHED{}
\def \articleTITLE{[Re] Three-dimensional wake topology and propulsive performance of low-aspect-ratio pitching-rolling plates}
\def \articleTYPE{Replication}
\def \articleDOMAIN{Computational Fluid Dynamics}
\def \articleBIBLIOGRAPHY{bibliography.bib}
\def \articleYEAR{2021}
\def \reviewURL{}
\def \articleABSTRACT{This article reports on a full replication study in computational fluid dynamics, using an immersed boundary method to obtain the flow around a pitching and rolling elliptical wing. As in the original study, the computational experiments investigate the wake topology and aerodynamic forces, looking at the effect of: Reynolds number (100--400), Strouhal number (0.4--1.2), aspect ratio, and rolling/pitching phase difference. We also include a grid-independence study (from 5 to 72 million grid cells). The trends in aerodynamic performance and the characteristics of the wake topology were replicated, despite some differences in results. We declare the replication successful, and make fully available all the digital artifacts and workflow definitions, including software build recipes and container images, as well as secondary data and post-processing code. Run times for each computational experiment on the nominal grid were between 8.1 and 13.8 hours to complete 5 flapping cycles, using two compute nodes with Dual 20-Core 3.70GHz Intel Xeon Gold 6148 CPUs and two NVIDIA V100 GPU devices each.}
\def \replicationCITE{Li, C., & Dong, H. (2016). Three-dimensional wake topology and propulsive performance of low-aspect-ratio pitching-rolling plates. Physics of Fluids, 28(7), 071901.}
\def \replicationBIB{li_dong_2016}
\def \replicationURL{https://doi.org/10.1063/1.4954505}
\def \replicationDOI{10.1063/1.4954505}
\def \contactNAME{Lorena A. Barba}
\def \contactEMAIL{labarba@gwu.edu}
\def \articleKEYWORDS{rescience c, Computational Fluid Dynamics, C++, Python, Singularity containers}
\def \journalNAME{ReScience C}
\def \journalVOLUME{4}
\def \journalISSUE{1}
\def \articleNUMBER{}
\def \articleDOI{}
\def \authorsFULL{Olivier Mesnard and Lorena A. Barba}
\def \authorsABBRV{O. Mesnard and L.A. Barba}
\def \authorsSHORT{Mesnard and Barba}
\title{\articleTITLE}
\date{}
\author[1,\orcid{0000-0001-5335-7853}]{Olivier Mesnard}
\author[1,\orcid{0000-0001-5812-2711}]{Lorena A. Barba}
\affil[1]{Department of Mechanical and Aerospace Engineering, The George Washington University, Washington, DC, USA}
