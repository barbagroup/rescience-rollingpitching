\section{Introduction}

The minimum requirements to conduct reproducible computational research is to make code and data available to others.
Reproducibility and replicability are essential components for the progress of evidence-based science.
Three years ago, the Congress directed the National Science Foundation to contract with the National Academies of Sciences, Engineering, and Medicine to assess reproducibility and replicability in scientific and engineering research.
A consensus report\supercite{nasem_2019} was released in May 2019 and provides findings and recommendations for improving rigor and transparency in research.
The report also provides clear definitions of ``reproducibility'' and ``replicability'' that are intended to apply across all fields of science:

\begin{itemize}
  \item[] \textbf{Reproducibility} is obtaining consistent results using the same input data; computational steps, methods, and code; and conditions of analysis.
  \item[] \textbf{Replicability} is obtaining consistent results across studies aimed at answering the same scientific question, each of which had obtained its own data. Two studies may be considered to have replicated if they obtain consistent results given the level of uncertainty inherent in the system under study.
\end{itemize}

\citet{peng_2011} introduced the idea of a reproducibility spectrum, in which he defines reproducible research as being a ``minimum standard for judging scientific claims when full independent replication of a study is not possible'' or not available.
The two extremes on the reproducibility spectrum are ``not reproducible'' (when the product of the study is only the a published manuscript) and "fully replicated" (the gold standard for a study).

In the past, we have attempted to fully replicate our own results.\supercite{krishnan_et_al_2014}
The outputs of this replication study have been published.\supercite{mesnard_barba_2017}
Knowing how challenging it was to replicate our own results, we now want to estimate how much effort it would take to replication the computational results from other research groups.
To partly answer that question, we have attempted to replicate a Computational Fluid Dynamics study on the pitching and rolling motion of an insect wing modeled as an elliptical flat plate.
The original study was published in 2016 by Li and co-authors.\supercite{li_dong_2016}
The output of the original study is a published manuscript.
The computational code, input data, and conditions of analysis that were used to produce the numerical results reported in the publications are not publicly available.
Thus, following Peng's reproducibility spectrum, we consider these studies to be not reproducible.
But can we replicate the scientific findings claimed in the publications?
And, can we do it in a reproducible manner?

The original study\supercite{li_dong_2016} deals with the rolling and pitching motion of an insect wing (modeled as an elliptical disk).
Many prior studies have focused on the pitching and/or heaving motion of three-dimensional wings.
But only few studies have focused the rolling and pitching motion.
The pitching-rolling wing has the potential to be a better canonical model for instigating the hydrodynamics of bio-inspired flapping propulsors.
The original study looked at the propulsive performance of a pitching and rolling elliptical disk over a wide range of parameters (Reynolds number, Strouhal, number, wing's aspect ratio, and phase-difference angle between the pitching and rolling).
They used their own research code to produce the numerical results; it solve the incompressible Navier-Stokes equations with a projection method and with an immersed boundary method to handle the motion of the wing on a fixed structured Cartesian grid.
The authors of the original publication have conducted a parametric study to investigate the effect of the Reynolds number, Strouhal number, and shape of the elliptical plate.
They used their own research code to produce the numerical results, which implements a different immersed-boundary method that those implemented in PetIBM.
The authors used a sharp-interface method (with a ghost-cell methodology), while we use a diffused-interface technique (with regularized delta functions) in PetIBM.

We have implemented the three-dimensional rolling and pitching kinematics in an open-source code that is version-controlled with Git and hosted on GitHub.
The research code is released under the permissive (non-copyleft) 3-Clause BSD license and uses the \href{https://github.com/barbagroup/PetIBM}{PetIBM} library\supercite{chuang_et_al_2018} to solve the equations.

\section{Numerical methods and implementation}

The present replication study models the wing with an elliptical flat plate.
The flat plate undergoes a rolling motion around the streamwise $x$-axis combined to a pitching motion around its spanwise axis.
The wing is characterized by its chord length $c$ and its spanwise length $S$.
The aspect ratio $AR$ of the wing is given by $AR = \frac{S^2}{A_\text{plan}}$, where $A_\text{plan}$ is the planform area of the plate ($A_\text{plan} = \frac{\pi c S}{4}$).

The rolling motion is defined by the instantaneous rolling position:

\begin{equation}
  \phi (t) = -A_\phi \cos \left( 2 \pi f t \right)
\end{equation}

where $t$ is the time, $f$ is the flapping frequency, and $A_\phi$ is the rolling amplitude.

The pitching motion along the (rolling) spanwise axis is defined by the instantaneous pitching position:

\begin{equation}
  \theta (t) = -A_\theta \cos(2 \pi f t + \psi)
\end{equation}

where $A_\theta$ is the pitching amplitude and $\psi$ is the phase-difference angle between the pitching and rolling motions.

In the present replication study, we generate the flow field variables (velocity and pressure) by numerically solving the three-dimensional Navier-Stokes for an incompressible viscous flow in a non-dimensional form.
The Reynolds number is defined as $Re = \frac{U_\infty c}{\nu}$, where $U_\infty$ is the incoming freestream speed and $\nu$ is the kinematic viscosity.
The convective and diffusion terms are advanced in time using second-order accurate Adams-Bashforth and Crank-Nicolson methods, respectively.
We impose a Dirichlet condition (streamwise velocity set to the freestream speed $U_\infty$) on all boundaries, except at the outlet where we use a convective boundary condition (to convey the vortical structures outside the computational domain).

Our code base, PetIBM, solves the incompressible Navier-Stokes equations via a projection method,seen as an approximate block-LU decomposition of the fully discretized equations.\supercite{perot_1993}
To compute the external flow around moving objects (such as a pitching-rolling wing), we use an immersed boundary technique.
In that framework, the fluid equations are solved over an extended domain that includes the inside of the immersed object.
This technique allows us to solve the equations on simple fixed structured Cartesian grids.
The boundary of the immersed objected is represented by a collection of Lagrangian markers at which locations we enforce a no-slip condition.
To take into account the presence of an immersed object, the fluid equations are modified in the vicinity of its boundary.
Different treatments lead to different immersed boundary methods.
The original study used a sharp-interface method with a ghost-cell methodology.\supercite{mittal_et_al_2008}
We aim to replicate the main scientific findings of the original study using a different immersed-boundary method.
In PetIBM, we employ regularized delta functions to transfer data between the Lagrangian markers and the Eulerian grid points (on which the fluid equations are solved).
PetIBM provide different implementations of the immersed boundary method, among them the immersed-boundary projection method\supercite{taira_colonius_2007} and its decoupled version.\supercite{li_et_al_2016}
(The later is used in the present replication study.)

Each time step, we sequentially solve three linear systems to compute an intermediate velocity field, the Lagrangian forces, and the pressure.
The system for the velocity is solved using a stabilized bi-conjugate gradient method (from PETSc library) with a Jacobi preconditioner with a convergence criterion based on the absolute $L_2$-norm of the residual and is set to $atol = 10^{-6}$.
We solve the system for the Lagrangian forces with a direct solver (SuperLU\_dist library).
The pressure Poisson system is solved with a conjugate-gradient method with a classical algebraic multigrid technique (from NVIDIA AmgX library); here too, the convergence is reached when the absolute $L_2$-norm of the residual is $10^{-6}$.

\begin{table}
  \centering
  \begin{tabular}{ll}
    \hline\hline
    Parameter & Values \\
    \hline
    Wing aspect ratio $AR$ & $1.27$, $1.91$, $2.55$ \\
    Reynolds number $Re$ & $100$, $200$, $400$ \\
    Strouhal number $St$ & $0.4$, $0.6$, $0.8$, $1.0$, $1.2$ \\
    Rolling amplitude $A_\phi$ & $45^o$ \\
    Pitching amplitude $A_\theta$ & $45^o$ \\
    Phase-difference angle $\psi$ & $60^o$, $70^o$, $80^o$, $90^o$, $100^o$, $110^o$, $120^o$ \\
    \hline\hline
  \end{tabular}
  \caption{Parameter values used in the present replication study.}
  \label{tab:parameters}
\end{table}

\section{Reproducible computational workflow}

The final product of the original study consists in a published manuscript in the journal Physics of Fluids.
Although the manuscript is well detailed, the code and input data used to produce the results were not made publicly available by the authors.
In that regards, we consider the original study to not be computationally reproducible.
Thus, we aim to replicate the scientific findings claimed in the original study with our own research software stack, and to produce reproducible results.

PetIBM\supercite{chuang_et_al_2018} is developed open-source, version-controlled with Git, and hosted on a public GitHub repository.
We release PetIBM under the permissive (non-copyleft) 3-Clause BSD license.
(Each release is also archived on the data repository Zenodo.)

Our implementation of the three-dimensional rolling and pitching wing with PetIBM is also open-source on GitHub and released under the same license, which allows others to use, modify, and re-distribute our work.
In addition, the version-controlled repository contains all the input data and processing scripts needed to reproduce the simulation results (from submitting and running the simulations to re-generating the figures of the present manuscript).
The repository also contains \textsc{README} files to guide readers that may be interested in re-running the analysis.
Upon acceptation of the present article, the application repository, as well as secondary data needed to reproduce the figures, have been archived on Zenodo with a digital-object identifier associated to it.

All simulations reported in the present article were submitted to our University-managed high-performance-computing (HPC) cluster called Pegasus.
(We used computational nodes with Dual 20-Core 3.70GHz Intel Xeon Gold 6148 processors and NVIDIA P100 GPU devices.)
To avoid the burden of building PetIBM and its applications on the cluster, we used the container technology from Docker and Singularity (citations needed).
Containers allow us the capture the conditions of analysis.
We have already used Docker containers in the past to create a reproducible workflow for scientific applications on the public cloud provider Microsoft Azure.\supercite{mesnard_barba_2020}
Here, we aim to adopt a similar workflow on our local HPC cluster.
In the early stage of this replication, we faced an obstacle: Docker is not available to users on Pegasus.
In fact, we learned that Docker is not available at most HPC centers for security reasons.
Indeed, submitting container-based jobs with Docker implies running a Docker daemon (a background process) that requires root privileges that users do not and should not have on shared production clusters.
Thus, we decided to adopt another container technology, Singularity, to conduct the replication study on Pegasus (citation needed).
Singularity is more recent than Docker and was designed from the ground up to prevent escalation of user privileges.
The technology has also the capability to convert Docker images into Singularity images.

Our reproducible workflow starts with creating a Docker image that installed PetIBM and its applications, as well as all there dependencies.
We pushed the image to a public registry on DockerHub.
Anyone interested in using the application code can now pull the image from the registry and create a container out of it with a faithfully reproduced computational environment.
Next, we use the cloud service Singularity Hub to build a Singularity image, which is then pulled via a command-line terminal and used to run container-based jobs on Pegasus.
Researchers interested in reproducing our results can pull our Singularity image to create a faithfully reproduced computational environment.

\section{Results}

\subsection{Grid-independence study}

In the original study, the authors reported the results of a grid-independence study to justify the spatial and temporal grid sizes used in the parametric study.
Force coefficients, profiles of the velocity components, profiles of the fluctuations of the kinetic energy, and distances between vortical structures in the near wake of the wing, were compared between the different grids to assess numerical independence in the results.
Here, we also report the results of our grid-independence study.

We use the same domain size as in the original study: $30c \times 25c \times 25c$.
The root of the wing (around which the plate undergoes the rolling and pitching motions) is located at the center of the computational domain.
The spatial grid is kept uniform with highest resolution in the region that covers the motion of the wing.
Outside this area, we also add a uniform layer in the sub-domain $\left[ -2c, 6c \right] \times \left[ -3c, 3c \right] \times \left[ -c, 2c \right]$, that covers the near-wake region, with grid-spacing size $\Delta x = 0.05c$.
(We opted for a smooth transition between the two uniform regions, in which the grid-cell widths are stretched with a constant ratio of $1.1$ in all directions, except in the streamwise direction behind the wing where we used a ratio of $1.03$.)
Finally, the grid-cell width is stretched to the external boundaries with a constant ration of $1.2$.
The wing is modeled as a flat ellipse with Lagrangian markers uniformly distributed on the surface (with a similar resolution as the grid-cell width of the background Eulerian grid).

For the present grid-independence study, we computed the flow with Reynolds number $Re = 200$, Strouhal number $St = 0.6$, and phase-difference angle $\psi = 90^o$.
We investigated the effect of the grid-cell width, the time-step size, and the exit criterion of the iterative solvers, on the numerical solution.
Moreover, since the original study modeled the wing with an elliptical disk (with a thickness equal to $3\%$ of the chord length), we also compared the results obtained with the flat plate to those using a similar disk.

\begin{table}
  \centering
  \begin{tabular}{lcccc}
    \hline\hline
    Grid & $(\Delta x / c)_\text{min}$ & $n_x \times n_y \times n_z$ & \# grid cells ($\times 10^6$) & \# body markers \\
    \hline
    Coarse & $0.03$ & $226 \times 185 \times 122$ & $5.1$ & $1005$ \\
    Nominal & $0.01$ & $331 \times 302 \times 211$ & $21.1$ & $9042$ \\
    Fine & $0.005$ & $467 \times 465 \times 333$ & $72.3$ & $36158$ \\
    \hline\hline
  \end{tabular}
  \caption{Characteristics of the computational grids used for in the grid-independence study. Body markers were uniformly distributed on the surface of the wing using a similar resolution as the background fluid grid.}
  \label{tab:independence_grid_charateristics}
\end{table}

To assess the effect of the grid-cell width on the solution, we computed five flapping cycles on three grids: coarse ($\Delta x = 0.03c$), nominal ($\Delta x = 0.01c$) and fine ($\Delta x = 0.005c$).
Table \ref{tab:independence_grid_charateristics} reports characteristics of the spatial grids used for the independence study.

\begin{figure}
  \centering
  \includegraphics[width=\textwidth]{../runs/independence/figures/force_coefficients_compare_dx.png}
  \caption{History of the thrust ($C_T$), lift ($C_L$), and spanwise ($C_Z$) coefficients over two flapping cycles of a circular plate ($AR = 1.27$) at Reynolds number $Re = 200$ and Strouhal number $St = 0.6$. We compare the solution obtained with PetIBM on three spatial grids: coarse ($\Delta x = 0.03c$), nominal ($\Delta x = 0.01c$), and fine ($\Delta x = 0.005c$). We also report the force coefficients published in \citet{li_dong_2016}.}
  \label{fig:independence_force_coefficients_dx}
\end{figure}

\begin{table}
  \centering
  \begin{tabular}{lcccc}
    \hline\hline
    Case & $\overline{C_T}$ & $\left( C_L \right)_\text{r.m.s.}$ & $\left( C_Z \right)_\text{r.m.s.}$ & $\eta$ \\
    \hline
    Li \& Dong (2016) & $1.462$ & $3.417$ & $1.912$ & $0.1927$ \\
    Coarse grid ($\Delta x = 0.03c$) & $0.443$ & $3.324$ & $2.098$ & $0.1002$ \\
    Nominal grid ($\Delta x = 0.01c$) & $0.914$ & $2.852$ & $1.651$ & $0.1635$ \\
    Fine grid ($\Delta x = 0.005c$) & $0.986$ & $2.817$ & $1.601$ & $0.1727$ \\
    Disk ($3\%$ thickness) & $0.762$ & $3.014$ & $1.886$ & $-$ \\
    $1000$ steps/cycle & $0.918$ & $2.853$ & $1.644$ & $0.1636$ \\
    Tighter solvers ($atol = 10^{-9}$) & $0.914$ & $2.852$ & $1.651$ & $0.1635$ \\
    \hline\hline
  \end{tabular}
  \caption{Results of the grid-independence study for a flapping wing with $AR = 1.27$, $St = 0.6$, $Re = 200$, and $\psi = 90^o$. We also report the results from \citet{li_dong_2016}.}
  \label{tab:independence_results}
\end{table}

Figure \ref{fig:independence_force_coefficients_dx} displays the history of the thrust, lift, and spanwise coefficients over two flapping cycles ($2000$ time steps per cycle), obtained on the coarse, nominal, finer spatial grids.
When using a similar grid resolution as in the original study ($\Delta x = 0.03c$), the instantaneous force components are noisy and we cannot distinguish two peaks in the lift coefficient, each half cycle.
Force productions are visually similar for the nominal and fine grids, with the presence of two peaks in the thrust and lift coefficients each half cycle (as reported in the original study).
Table \ref{tab:independence_results} reports hydrodynamic quantities and propulsive efficiency obtained from the simulations performed during the independence study.
Refining to computational grid leads to a relative difference of $7.8\%$ in the mean thrust coefficient, $1.2\%$ in the r.m.s. value for the lift coefficient, and $3\%$ in the r.m.s. value of spanwise coefficients.
Compared to \citet{li_dong_2016}, we obtain lower magnitude in the peaks of the force coefficients.
Note that we used the nominal spatial grid ($\Delta x = 0.01c$) to conduct a replication of the parametric study.

\begin{figure}
  \centering
  \includegraphics[width=\textwidth]{../runs/independence/figures/force_coefficients_compare_disk.png}
  \caption{History of the thrust ($C_T$), lift ($C_L$), and spanwise ($C_Z$) coefficients over two flapping cycles of a circular plate ($AR = 1.27$) at Reynolds number $Re = 200$ and Strouhal number $St = 0.6$. We compare solutions on the nominal grid where the wing is modeled as a flat plate and as a disk with a thickness of $3\%$ of the chord length (as done in the original study\supercite{li_dong_2016}).}
  \label{fig:independence_force_coefficients_disk}
\end{figure}

The original study modeled the wing with a disk (with a thickness equal to $3\%$ of the chord length).
Here, we use a different immersed boundary method which allows us to model thin volumes with flat surfaces.
Thus, we decided to employ a flat plate to model the wing, requiring fewer Lagrangian markers to discretize the immersed boundary.
Still, we wanted to compute the solution obtained with a disk (with same thickness as in the original study) to compare with the flat-plate-bases solution.
We obtained very similar history for the force coefficients (Figure \ref{fig:independence_force_coefficients_disk}), whether we used a disk or a flat plate.
Since using a disk does not really improve the solution, we decided to model the wing with a flat plate for the parametric study.

\begin{figure}
  \centering
  \includegraphics[width=\textwidth]{../runs/independence/figures/force_coefficients_compare_dt.png}
  \caption{History of the thrust ($C_T$), lift ($C_L$), and spanwise ($C_Z$) coefficients over two flapping cycles of a circular plate ($AR = 1.27$) at Reynolds number $Re = 200$ and Strouhal number $St = 0.6$. We compare the instantaneous force coefficients obtained with $n_t = 2000$ and $n_t = 1000$ time steps per flapping cycle obtained on the nominal grid. Markers show digitized data from \citet{li_dong_2016}.}
  \label{fig:independence_force_coefficients_dt}
\end{figure}

Although the authors of the original reported results to assess independence in the results when refining the temporal, they forgot to mention how may time steps per flapping cycle were computed.
(We know that force statistics and profiles of the velocity components remained unchanged when the time-step size was halved.)
Figure \ref{fig:independence_force_coefficients_dt} shows the history of the force coefficients obtained on two temporal grids ($2000$ and $1000$ time steps per flapping cycle), using the nominal spatial grid.
We note that the force production remain unchanged when doubling the time-step size.

Finally, we also checked that aerodynamic statistics do not change significantly when setting tighter convergence criterion for the iterative solvers.
Mean thrust coefficient, root-mean-square values of the lift and spanwise coefficients, and propulsive efficiency remain identical when decreasing the convergence criterion of the iterative solvers by 3 orders of magnitude (reported in Table \ref{tab:independence_results}).

\begin{figure}
  \centering
  \begin{subfigure}[c]{0.45\textwidth}
    \centering
    \includegraphics[width=\linewidth]{../runs/independence/figures/ux_profiles_compare_dx_dt.png}
    \caption{}
  \end{subfigure}
  \hfill
  \begin{subfigure}[c]{0.45\textwidth}
    \centering
    \includegraphics[width=\linewidth]{../runs/independence/figures/uy_profiles_compare_dx_dt.png}
    \caption{}
  \end{subfigure}
  \vspace{1cm}
  \begin{subfigure}[c]{0.45\textwidth}
    \centering
    \includegraphics[width=\linewidth]{../runs/independence/figures/uz_profiles_compare_dx_dt.png}
    \caption{}
  \end{subfigure}
  \hfill
  \begin{subfigure}[c]{0.45\textwidth}
    \centering
    \includegraphics[width=\linewidth]{../runs/independence/figures/kin_profiles_compare_dx_dt.png}
    \caption{}
  \end{subfigure}
  \caption{Comparison of the velocity profiles and profiles of the fluctuation of the kinetic energy in the wake of a flapping circular plate ($AR = 1.27$) at Reynolds number $Re = 200$ and Strouhal number $St = 0.6$. We report profiles at locations $x / c = 1$, $2$, $3$, $4$, and $5$. We compare the profiles obtained on the nominal grid, on the finer grid in space, and on the coarser grid in time. (a) Streamwise velocity ($\overline{u_1} - U_\infty$) profiles in the $x/y$ plane at $z = S / 2$; (b) transverse velocity ($\overline{u_2}$) profiles in the $x/y$ plane at $z = S / 2$; (c) spanwise velocity ($\overline{u_3}$) in the $x/z$ plane at $y = 0$; (d) fluctuation of the kinetic energy $( \overline{{u_1'}^2} + \overline{{u_2'}^2} + \overline{{u_3'}^2} ) / 2$ in the $x/y$ plane at $z = S / 2$. We also report data from the original study.\supercite{li_dong_2016}}
  \label{fig:independence_profiles}
\end{figure}

We also assess the effect of the numerical parameters (grid-spacing size and time-step size) on the flow features by comparing the profiles of the velocity components and of the fluctuations of the kinetic energy at several locations in the wake of the wing.
Figure \ref{fig:independence_profiles} reports the profiles with a direct comparison to the digitized data from \citet{li_dong_2016}.
We note that the profiles are visually similar when refining the spatial grid and when using fewer time steps per flapping cycle.

\begin{figure}
  \centering
  \begin{subfigure}[c]{0.45\textwidth}
    \centering
    \includegraphics[width=\linewidth]{../runs/independence/figures/wx_slice_c_distances.png}
    \caption{}
  \end{subfigure}
  \hfill
  \begin{subfigure}[c]{0.45\textwidth}
    \centering
    \includegraphics[width=\linewidth]{../runs/independence/figures/wx_slice_f_distances.png}
    \caption{}
  \end{subfigure}
  \caption{Two-dimensional contours of the streamwise vorticity ($-5 \leq w_x \leq 5$) in the $y/z$ plane at $t/T = 4.25$ in the near wake at $x / c = 0.3$ (a) and in the far wake at $x / c = 1.3$ (b). We compare the distances between vortical structures obtained on different grids to assess independence of the numerical solution.}
  \label{fig:independence_wx_distances}
\end{figure}

\begin{table}
  \centering
  \begin{tabular}{lcccc}
    \hline\hline
    Case & $d_1$ & $d_2$ & $d_3$ & $d_4$ \\
    \hline
    Nominal & $0.810$ & $0.630$ & $0.200$ & $0.900$ \\
    Finer in space & $0.780$ & $0.575$ & $0.215$ & $0.900$ \\
    Coarser in time & $0.810$ & $0.600$ & $0.190$ & $0.900$ \\
    \hline\hline
  \end{tabular}
  \caption{Comparison of the distances between vortex structures at $t / T = 4.25$ for a flapping wing with $AR = 1.27$, $St = 0.6$, $Re = 200$, and $\psi = 90^o$.}
  \label{tab:independence_wx_distances}
\end{table}

Based on the results of our independence study, we decided to replicate the parametric study using our nominal grid (with smallest cell-width $\Delta x = 0.01c$), computing $2000$ time steps per flapping cycles of a wing modeled as a flat plate.

\subsection{Baseline case}

As in the original study, we start by looking at the wake topology and aerodynamic forces generated by a flapping circular plate ($AR = 1.27$) at Reynolds number $Re = 200$, with Strouhal number $St = 0.6$ and a phase-difference angle of $\psi = 90^o$.

\begin{figure}
  \centering
  \begin{subfigure}[b]{0.3\textwidth}
    \centering
    \includegraphics[width=\linewidth]{../runs/Re200_St0.6_AR1.27_psi90/figures/qcrit_wx_perspective_zoom_view_0007750_post.png}
    \caption{$t / T = 3.875$}
  \end{subfigure}
  \hfill
  \begin{subfigure}[b]{0.3\textwidth}
    \centering
    \includegraphics[width=\linewidth]{../runs/Re200_St0.6_AR1.27_psi90/figures/qcrit_wx_perspective_zoom_view_0007875_post.png}
    \caption{$t / T = 3.9375$}
  \end{subfigure}
  \hfill
  \begin{subfigure}[b]{0.3\textwidth}
    \centering
    \includegraphics[width=\linewidth]{../runs/Re200_St0.6_AR1.27_psi90/figures/qcrit_wx_perspective_zoom_view_0008000_post.png}
    \caption{$t / T = 4.0$}
  \end{subfigure}
  \vspace{1cm}
  \begin{subfigure}[b]{0.3\textwidth}
    \centering
    \includegraphics[width=\linewidth]{../runs/Re200_St0.6_AR1.27_psi90/figures/qcrit_wx_perspective_zoom_view_0008250_post.png}
    \caption{$t / T = 4.125$}
  \end{subfigure}
  \hfill
  \begin{subfigure}[b]{0.3\textwidth}
    \centering
    \includegraphics[width=\linewidth]{../runs/Re200_St0.6_AR1.27_psi90/figures/qcrit_wx_perspective_zoom_view_0008375_post.png}
    \caption{$t / T = 4.1875$}
  \end{subfigure}
  \hfill
  \begin{subfigure}[b]{0.3\textwidth}
    \centering
    \includegraphics[width=\linewidth]{../runs/Re200_St0.6_AR1.27_psi90/figures/qcrit_wx_perspective_zoom_view_0008500_post.png}
    \caption{$t / T = 4.25$}
  \end{subfigure}
  \vspace{1cm}
  \begin{subfigure}[b]{0.3\textwidth}
    \centering
    \includegraphics[width=\linewidth]{../runs/Re200_St0.6_AR1.27_psi90/figures/qcrit_wx_perspective_zoom_view_0008625_post.png}
    \caption{$t / T = 4.3125$}
  \end{subfigure}
  \hfill
  \begin{subfigure}[b]{0.3\textwidth}
    \centering
    \includegraphics[width=\linewidth]{../runs/Re200_St0.6_AR1.27_psi90/figures/qcrit_wx_perspective_zoom_view_0008750_post.png}
    \caption{$t / T = 4.375$}
  \end{subfigure}
  \hfill
  \begin{subfigure}[b]{0.3\textwidth}
    \centering
    \includegraphics[width=\linewidth]{../runs/Re200_St0.6_AR1.27_psi90/figures/qcrit_wx_perspective_zoom_view_0008875_post.png}
    \caption{$t / T = 4.4375$}
  \end{subfigure}
  \caption{Wake topology for a flapping circular plate ($AR = 1.27$) with $St = 0.6$, $Re = 200$, and $\psi = 90^o$ at $t / T = 3.875$, $3.9375$, $4.0$, $4.125$, $4.1875$, $4.25$, $4.3125$, $4.375$, $4.4375$. We show the contours of the $Q$-criterion for $Q = 1$ (gray) and $Q = 6$ (colored by the streamwise vorticity, $-5 \leq w_x \leq 5$).}
  \label{fig:baseline_qcrit_perspective}
\end{figure}

\begin{figure}
  \centering
  \begin{minipage}{0.55\linewidth}
    \begin{subfigure}[t]{\linewidth}
      \includegraphics[width=\textwidth]{../runs/Re200_St0.6_AR1.27_psi90/figures/qcrit_wx_perspective_view_0008500_post.png}
      \caption{}
    \end{subfigure}
  \end{minipage}
  \begin{minipage}{0.35\linewidth}
    \begin{subfigure}[t]{\linewidth}
      \includegraphics[width=\textwidth]{../runs/Re200_St0.6_AR1.27_psi90/figures/qcrit_wx_lateral_view_0008500_post.png}
      \caption{}
    \end{subfigure}
    \vspace{1cm}
    \begin{subfigure}[b]{\linewidth}
      \includegraphics[width=\textwidth]{../runs/Re200_St0.6_AR1.27_psi90/figures/qcrit_wx_top_view_0008500_post.png}
      \caption{}
    \end{subfigure}
  \end{minipage}
  \caption{Wake topology of a pitching-rolling circular plate after four flapping cycles. We show flow structures  using the $Q$-criterion for $Q = 1$ (gray) and $Q = 6$ (colored by the streamwise vorticity, $-5 \leq w_x \leq 5$). (a) Perspective view, (b) side view, and (c) top view.}
  \label{fig:baseline_wake_topology}
\end{figure}

Figure \ref{fig:baseline_qcrit_perspective} shows the vortex shedding process in the vicinity of the wing at nine different time values.
We visualize vortex structures using the $Q$-criterion at $Q = 1$ (in gray) and $Q = 6$ (colored by the streamwise vorticity).
As the wing rolls downward (Figures \ref{fig:baseline_qcrit_perspective}(a-c)), we observe the formation of a ``C''-shaped vortex loop between the root vortex ($V_1$), the tip vortex ($V_2$), and the trailing-edge vortex (TEV).
When the wing reaches its lowest point (Figure \ref{fig:baseline_qcrit_perspective}(c)), we also note a strength asymmetry between $V_1$ and $V_2$.
As the plate starts rolling upward while pitching up (Figures \ref{fig:baseline_qcrit_perspective}(d-f)), we note the formation of new vortices from the trailing edge ($V_3$) and from the leading-edge vortex ($V_4$).
These vortices interact with each other to form a new ``C''-shaped vortex loop with opposite direction.
As the vortex loops move further downstream from the wing, they form a ``double-C''-shaped vortex structures, which was reported in the original study.
Each flapping cycle produces a pair of ``double-C''-shaped vortex structures (in opposite direction), leading to a bifurcated wake pattern (Figures \ref{fig:baseline_wake_topology}(a,b)).
The ``double-C''-shaped vortex structures evolve into single-loop vortex as they are convected downstream.
As noted in the original study, vortex rings shed in the wake gradually increase in size as they move further downstream, with a slight deflection in the spanwise $z$-direction towards the tip of the wing.

\begin{figure}
  \centering
  \begin{subfigure}[t]{0.24\textwidth}
    \centering
    \includegraphics[width=\linewidth]{../runs/Re200_St0.6_AR1.27_psi90/figures/wx_slice_a.png}
    \caption{}
  \end{subfigure}
  \begin{subfigure}[t]{0.24\textwidth}
    \centering
    \includegraphics[width=\linewidth]{../runs/Re200_St0.6_AR1.27_psi90/figures/wx_slice_b.png}
    \caption{}
  \end{subfigure}
  \begin{subfigure}[t]{0.24\textwidth}
    \centering
    \includegraphics[width=\linewidth]{../runs/Re200_St0.6_AR1.27_psi90/figures/wx_slice_c.png}
    \caption{}
  \end{subfigure}
  \begin{subfigure}[t]{0.24\textwidth}
    \centering
    \includegraphics[width=\linewidth]{../runs/Re200_St0.6_AR1.27_psi90/figures/wx_slice_d.png}
    \caption{}
  \end{subfigure}
  \vspace{0.5cm}
  \begin{subfigure}[t]{0.24\textwidth}
    \centering
    \includegraphics[width=\linewidth]{../runs/Re200_St0.6_AR1.27_psi90/figures/wx_slice_e.png}
    \caption{}
  \end{subfigure}
  \begin{subfigure}[t]{0.24\textwidth}
    \centering
    \includegraphics[width=\linewidth]{../runs/Re200_St0.6_AR1.27_psi90/figures/wx_slice_f.png}
    \caption{}
  \end{subfigure}
  \begin{subfigure}[t]{0.24\textwidth}
    \centering
    \includegraphics[width=\linewidth]{../runs/Re200_St0.6_AR1.27_psi90/figures/wx_slice_g.png}
    \caption{}
  \end{subfigure}
  \begin{subfigure}[t]{0.24\textwidth}
    \centering
    \includegraphics[width=\linewidth]{../runs/Re200_St0.6_AR1.27_psi90/figures/wx_slice_h.png}
    \caption{}
  \end{subfigure}
  \vspace{0.5cm}
  \begin{subfigure}[t]{0.24\textwidth}
    \centering
    \includegraphics[width=\linewidth]{../runs/Re200_St0.6_AR1.27_psi90/figures/wx_slice_i.png}
    \caption{}
  \end{subfigure}
  \begin{subfigure}[t]{0.24\textwidth}
    \centering
    \includegraphics[width=\linewidth]{../runs/Re200_St0.6_AR1.27_psi90/figures/wx_slice_j.png}
    \caption{}
  \end{subfigure}
  \begin{subfigure}[t]{0.24\textwidth}
    \centering
    \includegraphics[width=\linewidth]{../runs/Re200_St0.6_AR1.27_psi90/figures/wx_slice_k.png}
    \caption{}
  \end{subfigure}
  \begin{subfigure}[t]{0.24\textwidth}
    \centering
    \includegraphics[width=\linewidth]{../runs/Re200_St0.6_AR1.27_psi90/figures/wx_slice_l.png}
    \caption{}
  \end{subfigure}
  \caption{Two-dimensional slices of the streamwise vorticity ($-5 \leq w_x \leq 5$) at $t / T = 4.25$ in the $y/z$ plane at different $x$ locations in the wake of a flapping circular plate ($AR = 1.27$) with $St = 0.6$, $Re = 200$, and $\psi = 90^o$. (a-l)Wake locations: $x / c = 0$, $0.2$, $0.3$, $0.75$, $1.1$, $1.3$, $1.85$, $2.0$, $2.7$, $3.8$, $4.5$, and $5.25$.}
  \label{fig:baseline_wx_slices}
\end{figure}

Figure \ref{fig:baseline_wx_slices} shows two-dimensional slices of the streamwise vorticity component, in the $y/z$ plane at various locations along the $x$-axis, at the middle of the upstroke ($t/T = 4.25$)

\begin{figure}
  \centering
  \includegraphics[width=\linewidth]{../runs/Re200_St0.6_AR1.27_psi90/figures/force_coefficients.png}
  \caption{History of the thrust ($C_T$), lift ($C_L$), and spanwise ($C_Z$) coefficients during the fourth and fifth flapping cycles of a circular plate ($AR = 1.27$) with $St = 0.6$, $Re = 200$, and $\psi = 90^o$. We also report the digitized coefficients from \cite{li_dong_2016} and the instantaneous values of the instantaneous rolling and pitching angles (dashed blue line and dotted green line, respectively).}
  \label{fig:baseline_force_coefficients}
\end{figure}

\begin{table}
  \centering
  \begin{tabular}{lcccccc}
    \hline\hline
    Case & $\left| C_T \right|_\text{max}$ & $\overline{C_T}$ & $\left| C_L \right|_\text{max}$ & $\overline{C_L}$ & $\left| C_Z \right|_\text{max}$ & $\overline{C_Z}$ \\
    \hline
    Present & $2.66$ & $0.91$ & $3.96$ & $0.0$ & $2.56$ & $0.1$ \\
    \citet{li_dong_2016} & $3.98$ & $1.46$ & $5.35$ & $0.0$ & $3.19$ & $0.1$ \\
    \hline\hline
  \end{tabular}
  \caption{Statistics about force coefficients for the baseline case. We also report the values from \citet{li_dong_2016}.}
  \label{tab:baseline_force_coefficients}
\end{table}

Figure \ref{fig:baseline_force_coefficients} shows the history of the thrust, lift, and spanwise coefficients over two flapping cycles.
Table \ref{tab:baseline_force_coefficients} reports statistics about the force coefficients with comparison to the values reported in the original study.
Although we obtained smaller values in the peaks for the force coefficients as well as for the mean thrust coefficient, we still observe similar trends between our results and the one reported in \citet{li_dong_2016}.
Thrust peaks occur twice during each cycle, when the plate is near the center of its trajectory, with production of small drag when the plate starts to reverse its rolling direction.
Peak values of the lift and spanwise coefficients are in a similar range than the peak of the thrust coefficient.
Each half cycle, we also note the presence of two peaks in the thrust and lift coefficients, with the first peak having a smaller absolute value than the second one.

\subsection{Effect of the Strouhal number}

We now look at the effect of the Strouhal number on the wake topology and aerodynamic performances.
We computed the three-dimensional flow around a circular flat plate ($AR = 1.27$) at Reynolds number $200$ with a fixed phase-difference angle $\psi = 90^o$, while varying the Strouhal number $St$.

\begin{figure}
  \centering
  \begin{subfigure}[]{0.45\textwidth}
    \centering
    \includegraphics[width=\linewidth]{../runs/Re200_St0.4_AR1.27_psi90/figures/qcrit_wx_lateral_view_0008500_post.png}
    \caption{}
  \end{subfigure}
  \hfill
  \begin{subfigure}[]{0.45\textwidth}
    \centering
    \includegraphics[width=\linewidth]{../runs/Re200_St0.4_AR1.27_psi90/figures/qcrit_wx_top_view_0008500_post.png}
    \caption{}
  \end{subfigure}
  \vspace{1cm}
  \begin{subfigure}[]{0.45\textwidth}
    \centering
    \includegraphics[width=\linewidth]{../runs/Re200_St0.8_AR1.27_psi90/figures/qcrit_wx_lateral_view_0008500_post.png}
    \caption{}
  \end{subfigure}
  \hfill
  \begin{subfigure}[]{0.45\textwidth}
    \centering
    \includegraphics[width=\linewidth]{../runs/Re200_St0.8_AR1.27_psi90/figures/qcrit_wx_top_view_0008500_post.png}
    \caption{}
  \end{subfigure}
  \caption{Wake topology captured at $t / T = 4.25$ (when the plate is at middle point during the upstroke) for a circular plate ($AR = 1.27$) at Reynolds number $200$ and Strouhal numbers $0.4$ (top) and $0.8$ (bottom). Wake structures are represented with the Q-criterion for $Q = 1$ (gray) and $Q = 6$ (colored with the streamwise vorticity, $-5 \leq w_x \leq 5$). (a,c) lateral view of the wake; (b,d) top view.}
  \label{fig:strouhal_wake_topology}
\end{figure}

\begin{table}
  \centering
  \begin{tabular}{ccc}
    \hline\hline
    $St$ & $\alpha$ ($^o$) & $\beta$ ($^o$) \\
    \hline
    0.4 & 13 & 13 \\
    0.6 & 18 & 29 \\
    0.8 & 21 & 32 \\
    1.0 & 26 & 26 \\
    1.2 & 24 & 25 \\
    \hline\hline
  \end{tabular}
  \caption{Effect of the Strouhal number on the wake oblique angle ($\alpha$) and vortex ring orientation angle ($\beta$) at $Re = 200$ with $\psi = 90^o$.}
  \label{tab:strouhal_angles}
\end{table}

Figure \ref{fig:strouhal_wake_topology} shows lateral and top views of the shedding vortex pattern, at $t/T = 4.25$, obtained at Strouhal numbers $St = 0.4$ and $0.8$.
(Figure \ref{fig:baseline_wake_topology} shows the wake topology for $St = 0.6$.)
As reported in the original study, we note a decrease in the vorticity strength as the Strouhal number is lower, and a rapid evolution of the ``double-C''-shaped vortex structures into single vortex rings.
At higher Strouhal number ($St = 0.8$), we observe an enhancement of the mutual induction between two adjacent vortex rings.
\citet{li_dong_2016} reported values of the oblique angle ($\alpha$), defined as the angle between the horizontal $x$-axis and the line passing through the first two shed vortex rings adjacent to the trailing-edge of the plate.
They also evaluated the inclination angle ($\beta$) of a near vortex ring with respect to the wake centerline.
Table \ref{tab:strouhal_angles} reports the oblique and inclination angles obtained with our simulations.
Our angles are different from the ones reported in the original study.
The authors observed a monotonic increase in the oblique angle with respect to the Strouhal number and a peak in the orientation angle at Strouhal number $St = 1.0$ (followed by a sudden decrease).
Here, we observe an increase in the oblique angle but we a peak for Strouhal number $St = 1.0$, while the inclination peaks at Strouhal number $St = 0.8$.
Matching the observations reported in the original study, we also note the wake deflection along the mid-span axis (Figures \ref{fig:strouhal_wake_topology}(b, d) and Figure \ref{fig:baseline_wake_topology}(c)).
The main difference is that the authors reported that for the higher Strouhal number case, the wake starts deflecting towards the tip of the wing and then gradually deflects back in the far wake.
We do not observe the back deflection in our simulation at Strouhal number $St = 0.8$ (Figure \ref{fig:strouhal_wake_topology}(d)), which is probably due to the fact that we stopped the computation at $t/T = 5$, while it looks like the simulation in the original study was computed for a longer time (and therefore, more rings present in the wake).

\begin{figure}
  \centering
  \includegraphics[width=0.7\textwidth]{../runs/figures/efficiency_compare_St.png}
  \caption{Cycle-averaged thrust coefficient ($\overline{C_T}$) and propulsive efficiency ($\eta$) for the baseline case as functions of the Strouhal number. Data were computed and averaged during the fifth flapping cycle.}
  \label{fig:strouhal_propulsive_efficiency}
\end{figure}

\subsection{Effect of the Reynolds number}

We also computed the three-dimensional flow at additional Reynolds numbers $100$ and $400$ for a circular flat plate ($AR = 1.27$) with Strouhal number $St = 0.6$ and a $90$-degree phase difference between the rolling and pitching motions.

\begin{figure}
  \centering
  \begin{subfigure}[c]{0.45\textwidth}
    \centering
    \includegraphics[width=\linewidth]{../runs/Re100_St0.6_AR1.27_psi90/figures/qcrit_wx_perspective_zoom_view_0008500_post.png}
    \caption{}
  \end{subfigure}
  \hfill
  \begin{subfigure}[c]{0.45\textwidth}
    \centering
    \includegraphics[width=\linewidth]{../runs/Re400_St0.6_AR1.27_psi90/figures/qcrit_wx_perspective_zoom_view_0008500_post.png}
    \caption{}
  \end{subfigure}
  \caption{Vortex topology at $t / T = 4.25$ for a circular plate ($AR = 1.27$) with $St = 0.6$ and $\psi = 90^o$ at Reynolds numbers $Re = 100$ (a) and $Re = 400$ (b). We visualized vortical structures using the $Q$-criterion with $Q = 1$ (in gray) and $Q = 6$ (colored by the streamwise vorticity, $-5 \leq w_x \leq 5$).}
  \label{fig:reynolds_wake_topology}
\end{figure}

Figure \ref{fig:reynolds_wake_topology} shows a perspective view of the near-wake topology at $t/T = 4.25$ at Reynolds numbers $100$ and $400$, which can be compared to Figure \ref{fig:baseline_wake_topology}(f) for the case at $Re = 200$.
For all Reynolds numbers investigated, we note the formation of a double-loop vortex around the trailing edge of the plate.
As expected, vortex structures dissipate more rapidly for low Reynolds numbers.
At Reynolds number $Re = 400$, ``double-C''-shaped vortex rings propagate downstream.
Overall, we observe similar trends between our results and the one from the original study and note that the vortex dynamics remain unchanged for the range of Reynolds numbers investigated here.

\begin{figure}
  \centering
  \includegraphics[width=\linewidth]{../runs/figures/force_coefficients_compare_Re.png}
  \caption{History of the thrust ($C_T$), lift ($C_L$), and spanwise ($C_Z$) coefficients over two flapping cycles for a circular flat plate ($AR = 1.27$) at Reynolds numbers $100$, $200$, and $400$. Strouhal number is $0.6$ with a phase difference of $90^o$ for all these cases.}
  \label{fig:reynolds_force_coefficients}
\end{figure}

Figure \ref{fig:reynolds_force_coefficients} compares the history of the force coefficients over two flapping cycles for Reynolds numbers $Re = 100$, $200$, and $400$.
Although we obtained different statistics in the force coefficients (mean and peak values), we observe similar features as in the original study.
First, the absolute peak value and mean value for the thrust coefficient increase with the Reynolds number.
Second, we note the presence of two peaks in the lift coefficient every half cycle, for all Reynolds numbers.
Third, the negative peak in the spanwise coefficient becomes smaller as the Reynolds number decreases.
Finally, the mean lift and spanwise coefficients are approximately zero in all cases.

\subsection{Effect of the wing aspect ratio}

In the baseline simulation, the wing surface is circular with $AR = 1.27$.
We ran two additional simulations, with $AR = 1.91$ and $AR = 2.55$ to look at the effect of the wing aspect ratio on the wake topology and aerodynamic force coefficients.
(Other parameters remained identical compared to the baseline case.)

\begin{figure}
  \centering
  \begin{subfigure}[c]{0.45\textwidth}
    \centering
    \includegraphics[width=\linewidth]{../runs/Re200_St0.6_AR1.91_psi90/figures/qcrit_wx_perspective_zoom_view_0008500_post.png}
    \caption{}
  \end{subfigure}
  \hfill
  \begin{subfigure}[c]{0.45\textwidth}
    \centering
    \includegraphics[width=\linewidth]{../runs/Re200_St0.6_AR2.55_psi90/figures/qcrit_wx_perspective_zoom_view_0008500_post.png}
    \caption{}
  \end{subfigure}
  \caption{Vortex topology at $t / T = 4.25$ for an elliptical plate with $AR = 1.91$ (a) and $AR = 2.55$ (b) for $Re = 200$, $St = 0.6$, and $\psi = 90^o$. We visualized vortical structures using the $Q$-criterion with $Q = 1$ (in gray) and $Q = 6$ (colored by the streamwise vorticity, $-5 \leq w_x \leq 5$).}
  \label{fig:ratio_wake_topology}
\end{figure}

\begin{figure}
  \centering
  \begin{subfigure}[c]{0.3\textwidth}
    \centering
    \includegraphics[width=\linewidth]{../runs/Re200_St0.6_AR1.27_psi90/figures/wx_slice_yz_0008500.png}
    \caption{}
  \end{subfigure}
  \begin{subfigure}[c]{0.3\textwidth}
    \centering
    \includegraphics[width=\linewidth]{../runs/Re200_St0.6_AR1.91_psi90/figures/wx_slice_yz_0008500.png}
    \caption{}
  \end{subfigure}
  \begin{subfigure}[c]{0.34\textwidth}
    \centering
    \includegraphics[width=\linewidth]{../runs/Re200_St0.6_AR2.55_psi90/figures/wx_slice_yz_0008500.png}
    \caption{}
  \end{subfigure}
  \vspace{0.5cm}
  \begin{subfigure}[c]{0.32\textwidth}
    \centering
    \includegraphics[width=\linewidth]{../runs/Re200_St0.6_AR1.27_psi90/figures/wz_slice_xy_0008500.png}
    \caption{}
  \end{subfigure}
  \begin{subfigure}[c]{0.32\textwidth}
    \centering
    \includegraphics[width=\linewidth]{../runs/Re200_St0.6_AR1.91_psi90/figures/wz_slice_xy_0008500.png}
    \caption{}
  \end{subfigure}
  \begin{subfigure}[c]{0.32\textwidth}
    \centering
    \includegraphics[width=\linewidth]{../runs/Re200_St0.6_AR2.55_psi90/figures/wz_slice_xy_0008500.png}
    \caption{}
  \end{subfigure}
  \caption{Comparison of the two-dimensional contours of the vorticity at $t/T = 4.25$ for elliptical plates with $AR = 1.27$ (a, d), $AR = 1.91$ (b, e), and $AR = 2.55$ (c, f). (a-c): streamwise vorticity ($-5 \leq w_x \leq 5$) in the $y/z$ plane at $x/c = 0.3$ (near wake). (d-f): spanwise vorticity ($-5 \leq w_z \leq 5$) in the $x/y$ plane at $z = S/2$ (midspan).}
  \label{fig:ratio_vorticity_slices}
\end{figure}

Figure \ref{fig:ratio_wake_topology} shows snapshots of the vortical structures at $t/T = 4.25$ generated by elliptical plates with aspect ratios $AR = 1.91$ and $2.55$.
Figures \ref{fig:ratio_vorticity_slices} shows two-dimensional contours of the streamwise (a-c) and spanwise (d-f) vorticity components for different wing aspect ratios at the same time unit.
As reported in the original study, we too observe an increase in the size of the inner loop in streamwise direction (formed by vortices $V_3$ and $V_4$) along with the plate aspect ratio, and a decrease in the magnitude of the spanwise vorticity of the inner loop.

\begin{figure}
  \centering
  \includegraphics[width=\textwidth]{../runs/figures/force_coefficients_compare_AR.png}
  \caption{History of the thrust ($C_T$), lift ($C_L$), and spanwise ($C_Z$) coefficients over two flapping cycles for plates with different aspect ratios ($AR$) and with $St = 0.6$, $Re = 200$, and $\psi = 90^o$.}
  \label{fig:ratio_force_coefficients}
\end{figure}

Figure \ref{fig:ratio_force_coefficients} displays the history of the force coefficients over two flapping cycles, obtained with aspect ratios $AR = 1.27$, $1.91$, and $2.55$.
The mean thrust coefficient increases with the wing aspect ratio ($\overline{C_T} = 1.17$ for $AR = 1.91$ and $1.37$ for $AR = 2.55$).
We note a slight increase in the magnitude of the thrust peak as we increase the aspect ratio.
The authors of the original study reported a decrease in magnitude of the peak for the lift and spanwise coefficients.
However, our results show an increase in magnitude of the peak for the lift coefficient as the ratio increases.
Furthermore, the maximum value for the spanwise coefficient remains the same, while we note a decrease in the minimum value (as the ratio increases).

\subsection{Effect of the phase difference between pitching and rolling}

For all simulations reported so far, the phase-difference angle between the rolling and pitching motions was set to $\psi = 90^o$.
In this section, we look at the effect of the phase-difference angle on the wake topology and propulsive performance for a circular plate ($AR = 1.27$) at Reynolds number $Re = 200$ and Strouhal number $St = 0.6$.
We ran six additional simulations with phase-difference angle $\psi = 60^o$, $70^o$, $80^o$, $100^o$, $110^o$, and $120^o$.

\begin{figure}
  \centering
  \begin{subfigure}[]{0.45\textwidth}
    \centering
    \includegraphics[width=\linewidth]{../runs/Re200_St0.6_AR1.27_psi100/figures/qcrit_wx_lateral_view_0008500_post.png}
    \caption{}
  \end{subfigure}
  \hfill
  \begin{subfigure}[]{0.45\textwidth}
    \centering
    \includegraphics[width=\linewidth]{../runs/Re200_St0.6_AR1.27_psi100/figures/qcrit_wx_top_view_0008500_post.png}
    \caption{}
  \end{subfigure}
  \vspace{1cm}
  \begin{subfigure}[]{0.45\textwidth}
    \centering
    \includegraphics[width=\linewidth]{../runs/Re200_St0.6_AR1.27_psi110/figures/qcrit_wx_lateral_view_0008500_post.png}
    \caption{}
  \end{subfigure}
  \hfill
  \begin{subfigure}[]{0.45\textwidth}
    \centering
    \includegraphics[width=\linewidth]{../runs/Re200_St0.6_AR1.27_psi110/figures/qcrit_wx_top_view_0008500_post.png}
    \caption{}
  \end{subfigure}
  \vspace{1cm}
  \begin{subfigure}[]{0.45\textwidth}
    \centering
    \includegraphics[width=\linewidth]{../runs/Re200_St0.6_AR1.27_psi120/figures/qcrit_wx_lateral_view_0008500_post.png}
    \caption{}
  \end{subfigure}
  \hfill
  \begin{subfigure}[]{0.45\textwidth}
    \centering
    \includegraphics[width=\linewidth]{../runs/Re200_St0.6_AR1.27_psi120/figures/qcrit_wx_top_view_0008500_post.png}
    \caption{}
  \end{subfigure}
  \caption{Wake topology captured at $t / T = 4.25$ (when the plate is at middle point during the upstroke) for a circular plate ($AR = 1.27$) at Reynolds number $200$, Strouhal number $0.6$, and phase differences $\psi = 100^o$ (top), $110^o$ (center), and $120^o$ (bottom). Wake structures are represented with the $Q$-criterion for $Q = 1$ (gray) and $Q = 6$ (colored with the streamwise vorticity, $-5 \leq w_x \leq 5$). (a,c,e) Lateral view of the wake; (b,d,f) top view.}
  \label{fig:phase_wake_topology}
\end{figure}

\begin{table}
  \centering
  \begin{tabular}{ccc}
    \hline\hline
    $\psi$ ($^o$) & $\alpha$ ($^o$) & $\gamma$ ($^o$) \\
    \hline
    90 & 18 & -1 \\
    100 & 21 & -4 \\
    110 & 23 & -6 \\
    120 & 24 & -6 \\
    \hline\hline
  \end{tabular}
  \caption{Effect of the phase difference on the wake oblique angle ($\alpha$) and wake deflection angle ($\gamma$) at $St = 0.6$ and $Re = 200$.}
  \label{tab:phase_angles}
\end{table}

Figure \ref{fig:phase_wake_topology} shows snapshots (lateral and side views) of the wake topology at time $t/T = 4.25$ for phase-difference angles $\psi = 100^o$, $110^o$, and $120^o$.
(Figure \ref{fig:baseline_wake_topology} shows similar snapshots for the the baseline case with $\psi = 90^o$.)
Looking at the top views, we note an increase in the deflection angle $\gamma$ as the phase-difference angle increases.
In other words, larger phase-difference angles lead to larger deflection of the wake from the tip towards the root.
We also note the wake oblique angle $\alpha$ increases with the phase-difference angles.
Table \ref{tab:phase_angles} reports the wake oblique angles and wake deflection angles computed at time $t/T = 4.25$ for solutions obtained with various phase-difference angles.

\begin{figure}
  \centering
  \includegraphics[width=\textwidth]{../runs/figures/force_coefficients_compare_psi.png}
  \caption{History of the thrust ($C_T$), lift ($C_L$), and spanwise ($C_Z$) coefficients over two flapping cycles for a circular plate ($AR = 1.27$) for various phase-difference angles ($\psi$) between the pitching and rolling motions. Strouhal number is $0.6$ and Reynolds number is $200$ for all these cases.}
  \label{fig:phase_force_coefficients}
\end{figure}

\begin{figure}
  \centering
  \includegraphics[width=0.75\textwidth]{../runs/figures/efficiency_compare_psi.png}
  \caption{Cycle-averaged thrust coefficient ($\overline{C_T}$) and propulsive efficiency ($\eta$) for the baseline case as functions of the phase difference between the rolling and pitching motions. Data were computed and averaged during the fifth flapping cycle.}
  \label{fig:phase_efficiency}
\end{figure}

Figure \ref{fig:phase_force_coefficients} shows the history of the force coefficients over two flapping cycles for a circular wing with phase-difference angles between $90^o$ and $120^o$.
We note that peak magnitudes for all three force components increase with the phase-difference angle.
This increase coincides with the disappearance of the second peak in the thrust and lift forces, that was reported for the baseline case ($\psi = 90^o$).
Figure \ref{fig:phase_efficiency} reports the computed mean thrust coefficients and propulsive performance obtained for phase-difference angles between $60^o$ and $120^o$.
The mean thrust coefficient increases as we increase the phase-difference angle.
As reported in the original study, we obtained a maximum propulsive efficiency for the case with $\psi = 100^o$.

\section{Conclusion}

In this study, we attempted to replicate the scientific findings published in \citet{li_dong_2016}.
We used our own research software PetIBM\cite{chuang_et_al_2018} to conduct a similar parametric study.
Although the immersed boundary method used is different than the one used in the original study, we observe similar features in the force production, the wake topology, and the propulsive performance of a pitching and rolling wing.
Thus, we consider this replication attempt to be successful.

The final product of the original study consists in a published manuscript.
Thus, we consider the study to be not reproducible.
We made our best efforts to make our replication study to be reproducible.
Our computational application for the three-dimensional pitching-rolling motion of a wing (modeled as an elliptical flat plate) makes use of our open-source research software, called PetIBM.\cite{chuang_et_al_2018}
To increase the likeliness of the study to be reproducible, we created a GitHub repository for this study (\url{github.com/barbagroup/petibm-rollingpitching}).
The repository contains the source code of the PetIBM applications, as well as all input files of the simulations reported here and pre- and post-processing Python scripts.
We adopted a reproducible workflow to run computational simulations.
It makes use of Docker images and Singularity recipes to capture the computational environment.
With Singularity, we ran container-based jobs on our University-managed HPC cluster.
(SLURM job-submission scripts can be adapted to run on other platforms if readers are interested in reproducing our results.)
The repository is also archived on Zenodo, along with data needed to re-generate the figures on the present manuscript.
